\chapter{Glossary}

\begin{description}

\item[Channel] One of several data values persent in each pixel.
  Examples include red, green, blue, alpha, etc.  The data in one
  channel of a pixel may be represented by a single number, whereas the
  pixel as a whole requires one number for each channel.

\item[Client] A client (as in ``client application'') is a program or
  library that uses \product or any of its constituent libraries.

\item[Data format] The type of numerical representation used to
  store a piece of data.  Examples include 8-bit unsigned integers,
  32-bit floating-point numbers, etc.

\item[Image File Format] The specification and data layout of an
  image on disk.  For example, TIFF, JPEG/JFIF, OpenEXR, etc.

\item[Image Format Plugin] A DSO/DLL that implements the \ImageInput
  and \ImageOutput classes for a particular image file format.

\item[Format Plugin] See \emph{image format plugin}.

\item[Metadata] Data about data.  As used in \product, this means
  Information about an image, beyond describing the values of the pixels
  themselves.  Examples include the name of the artist that created the
  image, the date that an image was scanned, the camera settings used
  when a photograph was taken, etc.

\item[Native data format] The \emph{data format} used in the disk file
  representing an image.  Note that with \product, this may be different
  than the data format used by an application to store the image
  in the computer's RAM.

\item[Pixel] One pixel element of an image, consisting of one number
  describing each \emph{channel} of data at a particular location in an
  image.

\item[Plugin] See \emph{image format plugin}.

\item[Scanline] A single horizontal row of pixels of an image.  See also
  \emph{tile}.

\item[Scanline Image] An image whose data layout on disk is organized by
  breaking the image up into horizontal scanlines, typically with the
  ability to read or write an entire scanline at once.  See also
  \emph{tiled image}.

\item[Tile] A rectangular region of pixels of an image.  A rectangular
  tile is more spatially coherent than a scanline that stretches across
  the entire image --- that is, a pixel's neighbors are most likely in
  the same tile, whereas a pixel in a scanline image will typically have
  most of its immediate neighbors on different scanlines (requiring
  additional scanline reads in order to access them).

\item[Tiled Image] An image whose data layout on disk is organized by
  breaking the image up into rectangular regions of pixels called
  \emph{tiles}.  All the pixels in a tile can be read or written at
  once, and individual tiles may be read or written separately from
  other tiles.

\item[Volume Image] A 3-D set of pixels that has not only horizontal and
  vertical dimensions, but also a "depth" dimension.

\end{description}

\chapwidthend
